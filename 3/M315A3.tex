\documentclass[11pt,oneside,a4paper]{article}
\usepackage{mathtools}
\usepackage{multirow}
\usepackage{amssymb}
\newcommand{\iassv}{$\mathcal{I}$-assignment $v$ }
\newcommand{\ient}{\mathcal{I} \vDash_v}
\newcommand{\ii}{$\mathcal{I}$ }
\newcommand{\ck}{$\checkmark$}
\newcommand{\cs}{$\times$}
\begin{document}
\title{MATHS 315 Assignment 3}
\author{Jason Cairns - 8092261}
\maketitle

%q1
\section{}
\subsection*{(a)}
Let $\mathcal{I}$ be an interpretation with domain $D$, and let $v$ be an $\mathcal{I}$-assignment.\\
\\
\textbf{Case 1} \quad If $\mathcal{I} \nvDash_v \forall x \forall y \forall z Axyz$, then it is vacuously true that 
$$\mathcal{I} \vDash_v (\forall x \forall y \forall z Axyz \to \forall y \exists x \exists z Axyz) $$\\
\textbf{Case 2} \quad Suppose then that $\mathcal{I} \vDash_v \forall x \forall y \forall z Axyz$. Let $d \in D$.\\
Since $\mathcal{I} \vDash_v \forall x \forall y \forall z Axyz$, by definition 3.10, 7, for any $d_1, d_2, d_3 \in D$,
$$\mathcal{I} \vDash_{v \frac{d_1}{x} \frac{d_2}{y} \frac{d_3}{z}}Axyz$$
But $v\, \frac{d_1}{x} \frac{d_2}{y} \frac{d_3}{z} = v\, \frac{d_2}{y} \frac{d_1}{x} \frac{d_3}{z}$\\
\\
Since we have $D$ as a non-empty set, then there exists some $d_1$ and $d_3$, such that
$$\mathcal{I} \vDash_{v\, \frac{d_2}{y} \frac{d_1}{x} \frac{d_3}{z}}$$
So by definition 3.10,8, 
$$\mathcal{I} \vDash_{v \frac{d_2}{y}} \exists x \exists z Axzy$$
And since $d_2$ was chosen arbitrarily, we have that
$$\mathcal{I} \vDash_v \forall y \exists x \exists z Axyz$$
Therefore, by definition 3.10, 5, 
$$\mathcal{I} \vDash_v (\forall x \forall y \forall z Axyz \to \forall y \exists x \exists z Axyz)$$
Which, as $v$ and $\mathcal{I}$ were chosen arbitrarily, by definitions 3.12, 3.14, gives that
$$(\forall x \forall y \forall z Axyz \to \forall y \exists x \exists z Axyz)$$ is logically valid.

\subsection*{(b)}
The predicate form
$$(\forall x \exists y Bxy \to \forall y Byy)$$
is not logically valid.\\
\\
Let $\mathcal{I}$ be the interpretation with domain $\mathbb{N}$, with $B^\mathcal{I}(m,n)$ holding iff $m<n$. Let $v$ be an $\mathcal{I}$-assignment. For every $m \in \mathbb{N}$, $B^\mathcal{I}(m,m+1)$ holds, so for any $v$ we have $\mathcal{I} \vDash_{v \frac{m}{x} \frac{m+1}{y}} Bxy$. Thus by definition 3.10,8, $\mathcal{I} \vDash_v \forall x \exists y Bxy$.\\
There is no $n \in \mathbb{N}$ for which $B^\mathcal{I}(n,n)$ holds, so here is no $n$ for which $\mathcal{I} \vDash_{v \frac{n}{y}}Byy$, so by definition 3.10, 7, $\mathcal{I} \nvDash \forall Byy$. Therefore, by definition 3.10,5, $\mathcal{I} \nvDash_v (\forall x \exists y Bxy \to \forall y Byy)$. Thus by definitions 3.12, 3.14, $(\forall x \exists y Bxy \to \forall y Byy)$ is not logically valid.

\section{}
\subsection*{(a)}
\textbf{Case 1} \quad If $\mathcal{I} \nvDash A$, then by definition 3.10,5, $\mathcal{I} \vDash_v (A \to \exists A)$, and so $A \Rightarrow \exists A$.\\
\textbf{Case 2} \quad Suppose then that $\mathcal{I} \vDash A$. Then for every \iassv,  $\ient A$ in every interpretation \ii of $\mathcal{L}$, by definitions 3.12, 3.14. Consider $v(x)=d$ for some $d \in D$. Then we have $\mathcal{I}\vDash_{v \frac{d}{x}} A$ for some $d \in D$. As $\mathcal{I} \vDash A$, this is alid. Be definition 3.10, 8, and as $D$ is non-empty, this is equivalent to $\ient \exists x A$. Thus, $A \Rightarrow \exists A$.

\subsection*{(b)}
Let $A^\mathcal{I}(x)$ be the unary supremum predicate with domain $D$, that is, $\forall d \in D, \, d \leq x$.\\
Set $D$ to some finite partially ordered set, with cardinality $>$ 1.\\
We have for any \iassv that, based on the properties of $D$, $\mathcal{I} \vDash _{v \frac{e}{x}} Ax$ for some $e \in D$, as every finite, non-empty partially ordered set has some element that exists as the supremum.\\
Now, consider the \iassv such that, for all $d \in D$,
\[ v(x) =
  \begin{cases}
    \text{Sup($D$)}      & \quad \text{if } x \neq \text{Sup($D$)}\\
    \text{Inf($D$)}  & \quad \text{if } x = \text{ Sup($D$)}
  \end{cases}
\]
Therefore $\mathcal{I} \nvDash_v Ax$\\
Thus we have that $\exists x Ax \nRightarrow Ax$.

\section{}
\subsection*{(a)}
\subsubsection*{i}
$$(\exists z \forall x (Axy \to Bz) \vee \exists z Axz)\left(\frac{fzy}{x}\right) = (\exists z \forall x (Axy \to Bz) \vee \exists z Afzyz)$$
\subsubsection*{ii}$$((\exists x \forall y Axy \to Ayy) \to \forall y (Afxyz \wedge Bz))\left(\frac{fzy}{x}\right)$$$$ = (\exists x \forall y Axy \to Ayy) \to \forall y (Affzyyz \wedge Bz))$$

\subsection*{(b)}
\begin{center}
\begin{tabular}{|c|c|c|}
\hline
z for x & \cs & \ck \\
x for y & \cs & \cs \\
y for z & \ck & \cs \\
z for y & \cs & \ck \\
fxz for x & \cs & \ck \\
fygz for x & \cs & \cs \\
\hline
\end{tabular}
\end{center}

\section{}
\subsection*{(a)}
\begin{align*}
&1. \quad (B \to A) \quad \quad \quad \quad \quad \quad &\text{Hypothesis}\\
&2. \quad (A \to (\neg A \to \neg B) &\text{Tautology Instance}\\
&3. \quad (\neg A \to \neg B) &\text{1, 2, Modus Ponens}\\
&4. \quad ((\neg A \to \neg B) \to (( \neg A \to B) \to A)) &\text{K3}\\
&5. \quad ((\neg A \to B) \to A) &\text{3, 4, Modus Ponens}\\
&6. \quad \forall x ((\neg A \to B) \to A) &\text{5, Generalisation}
\end{align*}

\subsection*{(b)}
No - by Theorem 4.12 (The Deduction Theorem for $K_\mathcal{L}$), we can deduce $$\vdash_{K_\mathcal{L}} ((B \to A) \to \forall x ((\neg A \to B) \to A))$$ from $$\{ (B \to A) \} \vdash_{K_\mathcal{L}} \forall x ((\neg A \to B) \to A)$$ iff when generalisation is used, it is not used on a free variable. Here, generalisation was applied on $x$, which may be a free variable in $(B \to A)$.


\end{document}